\documentclass[11pt]{article}


\setlength{\parindent}{0pt}
\usepackage{xltxtra}
\usepackage{hyperref}
\hypersetup{hidelinks}
\usepackage{url}
\urlstyle{tt}
\usepackage{xcolor}
\definecolor{CVBlue}{RGB}{23,110,191}
\usepackage{calc}
\usepackage{graphicx}
\usepackage{tikz}
\usetikzlibrary{calc}
\usepackage{fontspec}
\usepackage{xeCJK}
\usepackage{enumitem}
\CJKsetecglue{} %% 取消中文与数字之间的间隙


%% 主文档字体设置
\setmainfont[
    Path = fonts/Main/,
    Extension = .otf,
    BoldFont = texgyretermes-bold.otf, % 加粗字体
]{texgyretermes-regular.otf} % 正文字体

% 中文字体设置
\setCJKmainfont[
    Path = fonts/hansans/,
    Extension = .ttf,
    BoldFont = NotoSansSC-Bold.ttf, % 加粗字体
]{NotoSansSC-Regular.otf} % 正文字体


\usepackage{relsize}
\usepackage{xspace}

% 使用 fontawesome(部分图标)
\usepackage{fontawesome} 

% A4纸,上下左右边距
\usepackage[
    a4paper,
    left=1.2cm,
    right=1.2cm,
    top=1.5cm,
    bottom=1cm,
    nohead
]{geometry}

\renewcommand{\baselinestretch}{1.5} % 行间距设为1.5

\usepackage{titlesec}
\usepackage{enumitem}
\setlist{noitemsep} % 取消列表项间的额外间距
%\setlist{nosep} % 取消所有垂直间距
\setlist[itemize]{topsep=0.25em, leftmargin=*}
\setlist[enumerate]{topsep=0.25em, leftmargin=*}

% --- 用于控制【不同项目之间】的垂直距离 ---
\newlength{\interProjectSpacing}
\setlength{\interProjectSpacing}{0.9em} % <--- 在此调整项目之间的距离
\newcommand{\projectsep}{\vspace{\interProjectSpacing}}

% --- 用于控制【项目标题】与下方【项目描述】的距离 ---
\newlength{\intraProjectTitleSep}
\setlength{\intraProjectTitleSep}{0.4em} % <--- 在此调整标题和描述的距离
\newcommand{\titlebreak}{\\[\intraProjectTitleSep]}

% --- 用于控制【项目描述】与下方【要点列表】的距离 ---
\newlength{\intraProjectListTopSep}
\setlength{\intraProjectListTopSep}{0.2em} % <--- 在此调整描述和列表的距离

% =======================================================================


\titleformat{\section}         % 定制 \section 命令 
{\large\bfseries\raggedright} % 将 section 标题设置为大号、粗体且左对齐
{}{0em}                      % 可用于为所有 section 添加前缀(如“章节...”)
{}                           % 可用于在标题前插入代码
[{\color{CVBlue}\titlerule}]  % 在标题后插入一条横线
\titlespacing*{\section}{0cm}{*1.6}{*1.2}



\begin{document}
\pagenumbering{gobble}

%%%% 利用tikz来定位照片
  %%%% 利用tikz来定位学校Logo,这里只在第一页显示
  \begin{tikzpicture}[remember picture, overlay] 
    \node[anchor = north west] at ($(current page.north west)+(0.5cm,+1.0cm)$) {\includegraphics[height=6cm]{zju.png}};
  \end{tikzpicture}%
\centerline{\LARGE\bfseries{王瑞}} 

\centerline{\normalsize{\faPhone\ 19705008739 \quad \faEnvelopeO\ \href{mailto:3240101517@zju.edu.cn}{3240101517@zju.edu.cn}}} 

\centerline{\normalsize{\faGithubSquare\ \href{https://github.com/zjuwangrui}{https://github.com/zjuwangrui}}} 
    
\section{\makebox[\widthof{\faGraduationCap}][c]{\color{CVBlue}\faGraduationCap}\ 教育背景}

\textbf{浙江大学} \hfill 2024.09 -- 至今

信息工程 本科(二年级)

\vspace{0.3em}
\textbf{GPA:}4.66/5.0(4.16/4.3,91.6/100)\quad 
\textbf{排名:}前5\%

\vspace{0.4em}
\textbf{核心课程:}
\begin{itemize}[nosep,leftmargin=1.6em]
    \item 线性代数  97/100(5.0/5.0)
    \item 微积分  97/100(5.0/5.0)
    \item 大学物理(甲)2  93/100(4.8/5.0)
    \item C语言程序设计基础及实验  92/100(4.8/5.0)
\end{itemize}


\section{\makebox[\widthof{\faUsers}][c]{\color{CVBlue}\faUsers}\ 项目经历}

% --- 第一个项目 ---
\textbf{Enroll - 学生选课助手 (浏览器插件)} 

项目描述:为解决学生选课过程中信息分散、操作繁琐的问题,与团队共同开发了服务于浙江大学学生的选课浏览器插件。该插件整合了课表导出、成绩查询、课堂录播下载等功能,提高学生选课与学习效率。
\begin{itemize}[nosep, topsep=\intraProjectListTopSep]
    \item \textbf{核心功能开发}:使用 \textbf{JavaScript} 与 \textbf{Chrome Extension API} 独立实现了课堂录播下载模块,包括视频资源识别、批量抓取与本地存储功能。
    \item \textbf{团队协作}:与多名同学合作开发,通过 \textbf{Git} 进行版本控制与代码管理,参与需求讨论、功能设计与测试优化全流程。
    \item \textbf{项目成果}:插件在浙江大学校内推广使用,获得良好反馈。掌握了浏览器插件开发流程与前端工程化协作经验。
    \item \textbf{相关链接}: GitHub仓库是\url{https://github.com/QSCTech/Enroll_nx}, 功能介绍推文是\url{https://mp.weixin.qq.com/s/OQEG9Q_kLU5yGWZcwtYdmA}
\end{itemize}

\projectsep

% --- 第二个项目 ---
\textbf{琼林宝阁 - 道教文化传播网站} 

项目描述:参与开发一个旨在合法传播道教文化资源的网站,提供道教经典、文章阅读、文化知识等内容,支持文章发布、搜索与分类浏览功能,促进传统文化数字化传播。
\begin{itemize}[nosep, topsep=\intraProjectListTopSep]
    \item \textbf{前端开发}:使用 \textbf{React.js} 框架,负责文章编辑与浏览模块的前端实现,包括富文本编辑、文章排版阅读功能。
    \item \textbf{协作开发}:参与项目需求分析、界面设计与功能测试,与后端同学协同完成数据接口对接。
    \item \textbf{相关链接}: 网站网址是\url{https://taoists.com.cn/}, GitHub仓库是\url{https://github.com/zjuatri/taoist_school_web}
\end{itemize}

\projectsep

% --- 第三个项目 ---
\textbf{五子棋对战网站} 

项目描述:为深入学习和实践Vue.js前端框架,独立设计并开发了一个完整的五子棋对战网站,实现了双人对战、胜负判定、棋局记录等核心功能。
\begin{itemize}[nosep, topsep=\intraProjectListTopSep]
    \item \textbf{前端实现}:使用 \textbf{Vue.js} 框架实现游戏界面、落子逻辑、胜负判定算法等核心功能。
    \item \textbf{状态管理}: 采用Vue的响应式系统管理游戏状态,确保界面与数据同步更新。
    \item \textbf{项目收获}: 通过完整项目开发,加深了对Vue组件化、数据绑定和前端工程化的理解。
    \item \textbf{相关链接}:  Github仓库是\url{https://github.com/zjuwangrui/wuziqi}
\end{itemize}

\section{\makebox[\widthof{\faCogs}][c]{\color{CVBlue}\faCogs}\ 技术栈}
\begin{itemize}[nosep]
    \item \textbf{编程语言:} JavaScript,  HTML/CSS
    \item \textbf{前端框架:} Vue.js, React.js
    \item \textbf{开发工具:} Git, Chrome DevTools, VS Code
    \item \textbf{其他技能:} 浏览器插件开发、RESTful API对接
\end{itemize}



\section{\makebox[\widthof{\faBullseye}][c]{\color{CVBlue}\faBullseye}\ 求职意向}
\begin{itemize}[nosep, topsep=\intraProjectListTopSep]
    \item \textbf{目标职位:} AI / Agent系统开发实习生 或 前后端开发实习生
    \item \textbf{工作方式:} 支持远程协作,每周可投入≥25小时
    \item \textbf{期望收获:} 希望参与真实系统迭代,提升工程部署、系统设计与AI应用能力
\end{itemize}
    

\section{\makebox[\widthof{\faLightbulbO}][c]{\color{CVBlue}\faLightbulbO}\ 个人优势与匹配度}
\begin{itemize}[nosep, topsep=\intraProjectListTopSep]
    \item \textbf{自驱力强:} 所有项目均为主动发起/参与,能独立推进任务并持续迭代
    \item \textbf{丰富开发经验:} 具备基础的Git工作流、团队协作、前端开发经验
    \item \textbf{学习适应快:} 大部分项目边做边学,适应能力强。能快速掌握新技术并应用于实践
\end{itemize}


\end{document}